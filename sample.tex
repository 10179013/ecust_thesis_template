\documentclass[UTF8,hyperref]{pkuthss}

% 产生 originauth.tex 里的 \Square。
\usepackage{wasysym}
% 提供 verbatiminput 命令和 comment 环境。
\usepackage{verbatim}

% 设置页芯居中。
\geometry{centering}
% 设定行距。
\renewcommand{\baselinestretch}{1.5}

% 使引用标记成为上标。
\newcommand{\supercite}[1]{\textsuperscript{\cite{#1}}}
% 罗列环境中如果每个项目都只有一行左右,则会显得很松散,此时可采用这个命令。
\newcommand{\denseenum}{\setlength{\itemsep}{0pt}}

\begin{document}
	% 各种文档信息。
	\renewcommand{\thesisname}{本科生毕业论文}
	% 题目一般不宜超过 20 个字。
	\title{北京大学论文文档模板\\v1.2 beta}
	\etitle{The PKU dissertation document class\\v1.2 beta}
	\author{盖茨波$\cdot$钛$\cdot$维克托}
	\eauthor{Casper Ti.\ Vector}
	\studentid{0XXXXXXX}
	\date{二〇一〇年七月}
	\school{化学与分子工程学院}
	\major{化学}
	\emajor{Chemistry}
	\direction{理论和计算化学}
	\mentor{XX~教授}
	\ementor{Prof.\ XX}
	% 关键词应有 3~5 个。
	\keywords{\LaTeX2e,排版,文档类,\CTeX}
	\ekeywords{\LaTeX2e, typesetting, document class, \CTeX}

	%% 以下为正文之前的部分,页码为小写罗马数字,但不显示页眉和页脚。
	\frontmatter\pagenumbering{roman}\pagestyle{empty}

	\maketitle
	% 版权声明。
	\include{chap/copyright}
	% 中英文摘要。
	\include{chap/abstract}
	% 自动生成目录。
	\tableofcontents

	%% 以下为正文,页码为小写罗马数字,但不显示页眉和页脚。
	\mainmatter\pagenumbering{arabic}\pagestyle{fancy}

	% 绪言。
	\include{chap/introduction}
	% 各章节。
	\include{chap/chap1}
	\include{chap/chap2}
	\include{chap/chap3}
	\include{chap/chap4}
	% 结论。
	\include{chap/conclusion}

	\begin{appendix}
		% 参考文献。
		\bibliographystyle{chinesebst}\bibliography{sample}
		% 此命令手动地在目录中增加相当于章级别的一行。
		\addcontentsline{toc}{chapter}{参考文献}
		% 此命令和真实的一级章命令结合,从而使 \addcontentsline
		% 在目录中产生的页码正常。
		\phantomsection

		% 各附录。
		\include{chap/encl1}
	\end{appendix}

	%% 以下为正文之后的部分,页码为大写罗马数字。
	\backmatter\pagenumbering{Roman}

	% 致谢。
	\include{chap/thanks}
	% 原创性声明和使用授权说明,不显示页码。
	\pagestyle{empty}
	\include{chap/originauth}
\end{document}

