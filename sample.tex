\documentclass[a4paper]{ecust_thesis_translation}

% !!非常重要!!
% 这份模版使用了很多xeTeX的宏包,如xeXJK,fontspec等,因此:
% **请使用xeTeX编译**


% 设定文献翻译相关信息
% eAuthor=英文原作者
% cStudent=中文翻译者(学生)
% cTitle=论文中文标题
% cKeywords=论文中文关键字
\author{王\quad 翔}
\class{电自 091}
\studentNo{10094309}
\title{一种基于Boosting的可用于脑机接口的P300检测方法}
\cKeywords{Boosting, 脑机接口, 自发脑电, P300, 最小二乘法}
\eKeywords{Boosting, BCI, EEG, P300}


% 可以按照个人的喜好自行设置字体,默认字体为
% - 英文=Liberation Serif;
% - 宋体=AR PL UMing CN(简体宋体);
% - 黑体=WenQuanYi Zen Hei(文泉驿正黑);
% 具体请参考ecusttrans.cls:23-29

% 自行设置字体可参考此处
\setmainfont{Liberation Serif}
\setCJKmainfont[BoldFont={FZYaSong-H-GBK}]{FZYaSong-M-GBK}
\setCJKfamilyfont{hei}[BoldFont=FZLanTingHei-B-GBK]{FZLanTingHei-R-GBK}
% 请注意:很多中文字体没有粗体的概念,因此在TeX中简单地使用\textbf是
% 不会出现粗体的(摊手),最好使用有独立的粗体和斜体的中文字体。如果使
% 用的字体粗体和斜体是单独的名称,请参考下面格式进行引用。

\newcommand\unsure{\textit}

\begin{document}
% 设置中文缩进
\setlength{\parindent}{2em}
% 设置标点符号挤压模式为半角,可选项有半角、全角、行末半角和开明
\punctstyle{banjiao}

  % 模版使用形式和普通article类一样
  % - 使用\maketitle命令显示标题;
  % - 使用\section{}命令开始一个章节;
  % - 使用\subsection{}开始一个子章节;
  % - 使用\subsubsection{}开始一个孙章节;

  \maketitle

  % - 使用\begin{cAbstract} 和 end{cAbstract} 来开始和结束中文摘要
  %   模版会自动加载关键词。
  \begin{abstract}[chinese]
  Gradient boosting 是一类通过多个弱分类器来建立强分类器的机器学习算法。本文主要描述了一种基于 gradient boosting 用于检测自发脑电(EEG)中的事件相关电位(ERP)的 gradient boosting 算法。该算法用于检测人的自发脑电中的 P300 信号,并将检测到的信号作为一个拼写装置的原始信号。本文所述方法的关键特性在于其分类的准确性及其概念的简单性。本算法经过一组由本实验室记录的数据集以及一组2003年BCI竞赛中所用到的测试数据集的测试。P300 拼写范式实验中的准确率在90\%到100\%之间。尤为突出的是,本算法完全正确地完成了所有出自 BCI 竞赛测试数据集的字符推断任务。
  \end{abstract}

  \section{导论}

  P300是一种人类自发脑电中的特征波形,一般作为对小概率刺激事件的响应出现。经典的 Oddball 范式实验常常用来诱发P300:呈现给被试的是随机顺序的两类刺激,其中靶刺激为小概率事件,被试则被要求判断刺激是否为靶刺激。

  L.A.Farwell 和 E.Donchin 曾首创性地用 Oddball 范式来进行脑机接口实验。在实验过程中,一个装有字母符号的6x6的矩阵被呈现给用户,该矩阵的行和列会按照随机顺序点亮。被试可以通过默记目标字符的点亮次数来从矩阵中选择一个字符。目标字符每点亮一次,就会诱发一个 P300 波,使用特定的算法就可以检测到该 P300 波形。

  本文阐述了一种简单却强大的方法来从自发脑电中检测 P300,并用该方法来建立一个基于 P300 的拼写装置。我们使用了 gradient boosting 和最小二乘法相结合的方式来建立P300检测器。

  Gradient boosting 和最小二乘法的组合是一种有意思的检测 P300 的方法,因为这对组合有如下特性:

  \begin{itemize}
    \item 该算法通过一种朴素的方式建立线性分类规则,因此将分类器套用到新的数据上只需要少量操作,且可以实时地对 EEG 进行分类操作。此外,分类规则也很非常易于理解,也就是说,从分类规则可以很容易地看出哪些样本和哪些通道对于检测 P300 比较重要。
    \item 在分类准确度上,本文所所述方法和当前最先进的方法相比效果更为显著。以2003年 BCI 竞赛的测试数据为例,gradient boosting 略优于优胜方案。
    \item 复杂的优化算法,例如支持向量机或者独立组件分析方法,对本文所述方法而言都不是必需的。因此本算法具有易于实现、使用和扩展的特性。
  \end{itemize}

  本文余下部分的主要安排如下:第二节主要阐述本实验所使用的实验范式、被试和数据处理;第三节主要阐述 boosting 算法;第四节则主要阐述分类器的输出是如何用于推断被试选择的字符的;实验结果呈现在第五节;第六节对全文做了总结。

  \section{被试和方法}
  \subsection{被试}

  一位19年前颈脊椎受过损伤\footnote{原文为 complete cervical spinal cord injury。}的男性被试(被试 S1)和一位健康的男性被试(被试 S2)参与了实验。两位被试分别在36岁和28岁有过BCI验的经历。

  \subsection{实验范式}

  \unsure{我们使用了类似于文献[1]中所述的方法来记录和标机数据,展示给被试的是一个显示在笔记本显示器上的装有字母和数字1-9的6x6矩阵。}该矩阵的行和列以100毫秒的间隔交替闪烁,且矩阵的闪烁经过\unsure{block randomize}处理,也就是说,经过12次高亮闪烁后,每一行和每一列都恰好被点亮了一次。
\end{document}
