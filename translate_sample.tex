\documentclass[a4paper]{ecust_translation}

% !!非常重要!!
% 这份模版使用了很多xeTeX的宏包,如xeXJK,fontspec等,因此:
% **请使用xeTeX编译**

% 如果不希望使用超链接(例如sample中的两个脚注的超链接效果),请
% 把这句话删除。
\usepackage{hyperref}

% 设定文献翻译相关信息
% eAuthor=英文原作者
% cStudent=中文翻译者(学生)
% cTitle=论文中文标题
% cKeywords=论文中文关键字
\eAuthor{Ulrich Hoffman, Gary Gacia, Jean-Marc Vesin, Karin Diserens and Touradj Ebrahimi}
\cStudent{王\quad 翔}
\cStudentClass{电自\space 091}
\cStudentNo{10094309}
\cTitle{一种基于Boosting的可用于脑机接口的P300检测方法}
\cKeywords{Boosting, 脑机接口, 脑电图, P300, 最小二乘法}

% 设置页面样式为 translation(自定义样式)
% 具体请参考ecusttrans.cls:61-67
\pagestyle{translation}

% 可以按照个人的喜好自行设置字体,默认字体为
% - 英文=Liberation Serif;
% - 宋体=AR PL UMing CN(简体宋体);
% - 黑体=WenQuanYi Zen Hei(文泉驿正黑);
% 具体请参考ecusttrans.cls:23-29

% 自行设置字体可参考此处
\setmainfont{Liberation Sans}
% !请注意:很多中文字体没有粗体的概念,因此在TeX中简单地使用\textbf是
% 不会出现粗体的(摊手),最好使用有独立的粗体和斜体的中文字体。如果使
% 用的字体粗体和斜体是单独的名称,请参考下面格式进行引用。
% \setCJKmainfont[BoldFont={FZYaSong-H-GBK}]{FZYaSong-M-GBK}
% \setCJKfamilyfont{hei}[BoldFont=FZLanTingHei-B-GBK]{FZLanTingHei-R-GBK}

\begin{document}
  % 设置中文缩进
  \setlength{\parindent}{2em}
  % 设置标点符号挤压模式为半角,可选项有半角、全角、行末半角和开明
  \punctstyle{banjiao}

  % 模版使用形式和普通article类一样
  % - 使用\maketitle命令显示标题;
  % - 使用\section{}命令开始一个章节;
  % - 使用\subsection{}开始一个子章节;
  % - 使用\subsubsection{}开始一个孙章节;
  \maketitle

  % - 使用\begin{cAbstract} 和 end{cAbstract} 来开始和结束中文摘要
  %   模版会自动加载关键词。
  \begin{cAbstract}

  Gradient boosting 是一种基于许多弱分类器来建立一个强分类器的机器学习算法。本文主要阐述了一种建立在 gradient boosting 之上的,用于检测脑电图(EEG)中的事件相关电位(ERP)的算法。该算法通过检测人类脑电图中的P300电位来建立一种脑机接口(BCI)\footnote{也有翻译为{“自发电位”}的},一个拼写装置。本文所述方法的重要特性在于其分类的准确性及其概念的简单性。本算法经过一组由本实验室记录的数据集以及一组2003年BCI竞赛中所用到的测试数据集的测试。P300 拼写范式实验中的准确率在90\%到100\%之间。值得注意的是,本算法完全正确地完成了所有出自BCI竞赛测试数据集的字符推断任务。

  \end{cAbstract}

  \wuhao
  \section{导论}
  P300是一种人类脑电图中的特征波形,一般作为对小概率刺激事件的响应出现。经典的 Oddball 范式实验常常用来诱发P300:呈现给被试的是随机顺序的两类刺激,其中一类刺激为小概率事件,被试则被要求判断刺激属于哪一类。

  L.A.Farwell 和 E.Donchin 曾首创性地用 Oddball 范式实验来建立脑机接口。他们所用的方法中,呈现给用户的是一个装有字符的6x6的矩阵,该矩阵的行和列会按照随机顺序点亮。被试可以通过计数目标字符的点亮此书来从矩阵中选择一个字符。目标字符每点亮一次,就诱发一个 P300 波形,该波形可以通过适当的算法检测到。

  本文阐述了一种简单却强大的方法来从脑电图中检测 P300,并用该方法来建立一个基于 P300 的拼写装置。我们使用了 gradient boosting 和最小二乘法相结合的方式来建立P300检测器。

  Gradient boosting 和最小二乘法的组合是一种有意思的检测 P300 的方法,因为这对组合有如下特性:

  \begin{itemize}
    \item 该算法通过一种朴素的方式建立线性分类规则,因此将分类器套用到新的数据上只需要少量操作,且可以实时地对 EEG 进行分类操作。此外,分类规则也很非常易于理解,也就是说,从分类规则可以很容易地看出哪些样本和哪些通道对于检测 P300 比较重要。
    \item 在分类准确度上,本文所所述方法和当前最先进的方法相比效果更为显著。以2003年 BCI 竞赛的测试数据为例,gradient boosting 略优于优胜方案。
    \item 复杂的优化算法,例如支持向量机或者独立组件分析方法,对本文所述方法而言都不是必需的。因此本算法具有易于实现、使用和扩展的特性。
  \end{itemize}

  本文余下部分的主要安排如下:第二节主要阐述本实验所使用的实验范式、被试和数据处理;第三节主要阐述 boosting 算法;第四节则主要阐述分类器的输出是如何用于推断被试选择的字符的;实验结果呈现在第五节;第六节对全文做了总结。
  \section{被试和方法}
  \subsection{被试}
  一位19年前颈脊椎受过损伤\footnote{原文为~complete cervical spinal cord injury。}的男性被试(被试~S1)和一位健康的男性被试(被试~S2)参与了实验。两位被试分别在36岁和28岁有过BCI验的经历。
  \subsection{实验范式}

  \begin{equation}
  s_1, s_2= - \frac{B}{2G} \sqrt{(\frac{B}{2G})^2 - \frac{B}{C}}
  \end{equation}
\end{document}
