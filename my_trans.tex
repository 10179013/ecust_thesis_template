\documentclass[a4paper]{ecust_thesis_translation}

% !!非常重要!!
% 这份模版使用了很多xeTeX的宏包,如xeXJK,fontspec等,因此:
% **请使用xeTeX编译**


% 设定文献翻译相关信息
% eAuthor=英文原作者
% cStudent=中文翻译者(学生)
% cTitle=论文中文标题
% cKeywords=论文中文关键字
\author{王\ \ 翔}
\class{电自 091}
\studentNo{10094309}
\title{一种基于Boosting的脑机接口 P300 检测方法}
\cKeywords{Boosting, 脑机接口, 自发脑电, P300, 最小二乘法}
\eKeywords{Boosting, BCI, EEG, P300}


% 可以按照个人的喜好自行设置字体,默认字体为
% - 英文=Liberation Serif;
% - 宋体=AR PL UMing CN(简体宋体);
% - 黑体=WenQuanYi Zen Hei(文泉驿正黑);
% 具体请参考ecusttrans.cls:23-29

% 自行设置字体可参考此处
\setmainfont{Times New Roman}
\setCJKmainfont[BoldFont={FZYaSong-M-GBK}]{SimSun}
\setCJKfamilyfont{hei}[BoldFont={FZLanTingHei-B-GBK}]{SimHei}
% 请注意:很多中文字体没有粗体的概念,因此在TeX中简单地使用\textbf是
% 不会出现粗体的(摊手),最好使用有独立的粗体和斜体的中文字体。如果使
% 用的字体粗体和斜体是单独的名称,请参考上面格式进行引用。

\renewcommand\![1]{\immature{#1}}

\newcommand\unsure{\textit}

\begin{document}
% 设置中文缩进
\setlength{\parindent}{2em}
% 设置标点符号挤压模式为半角,可选项有半角、全角、行末半角和开明
\punctstyle{banjiao}

% 模版使用形式和普通article类一样
% - 使用\maketitle命令显示标题;
% - 使用\section{}命令开始一个章节;
% - 使用\subsection{}开始一个子章节;
% - 使用\subsubsection{}开始一个孙章节;

\maketitle

% - 使用\begin{cAbstract} 和 end{cAbstract} 来开始和结束中文摘要
%   模版会自动加载关键词。
\begin{abstract}[chinese]
Gradient boosting 是一类通过多个弱分类器来建立强分类器的机器学习算法。本文主要描述了一种基于 gradient boosting 用于检测自发脑电(EEG)中的事件相关电位(ERP)的 gradient boosting 算法。该算法用于检测人的自发脑电中的 P300 信号,并将检测到的信号作为一个拼写装置的原始信号。本文所述方法的关键特性在于其分类的准确性及其概念的简单性。本算法经过一组由本实验室记录的数据集以及一组2003年BCI竞赛中所用到的测试数据集的测试。P300 拼写范式实验中的准确率在90\%到100\%之间。尤为突出的是,本算法完全正确地完成了所有出自 BCI 竞赛测试数据集的字符推断任务。
\end{abstract}

\section{导论}

P300是一种人类自发脑电中的特征波形,一般作为对小概率刺激事件的响应出现。经典的 Oddball 范式实验常常用来诱发P300:呈现给被试的是随机顺序的两类刺激,其中靶刺激为小概率事件,被试则被要求判断刺激是否为靶刺激。

L.A.Farwell 和 E.Donchin 曾首创性地用 Oddball 范式来进行脑机接口实验。在实验过程中,一个装有字母符号的6x6的矩阵被呈现给用户,该矩阵的行和列会按照随机顺序点亮。被试可以通过默记目标字符的点亮次数来从矩阵中选择一个字符。目标字符每点亮一次,就会诱发一个 P300 波,使用特定的算法就可以检测到该 P300 波形。

本文阐述了一种简单却强大的方法来从自发脑电中检测 P300,并用该方法来建立一个基于 P300 的拼写装置。我们使用了 gradient boosting 和最小二乘法相结合的方式来建立P300检测器。

Gradient boosting 和最小二乘法的组合是一种有意思的检测 P300 的方法,因为这对组合有如下特性:

\begin{itemize}
  \item 该算法通过一种朴素的方式建立线性分类规则,因此将分类器套用到新的数据上只需要少量操作,且可以实时地对 EEG 进行分类操作。此外,分类规则也很非常易于理解,也就是说,从分类规则可以很容易地看出哪些样本和哪些通道对于检测 P300 比较重要。
  \item 在分类准确度上,本文所所述方法和当前最先进的方法相比效果更为显著。以2003年 BCI 竞赛的测试数据为例,gradient boosting 略优于优胜方案。
  \item 复杂的优化算法,例如支持向量机或者独立组件分析方法,对本文所述方法而言都不是必需的。因此本算法具有易于实现、使用和扩展的特性。
\end{itemize}

本文余下部分的主要安排如下:第二节主要阐述本实验所使用的实验范式、被试和数据处理;第三节主要阐述 boosting 算法;第四节则主要阐述分类器的输出是如何用于推断被试选择的字符的;实验结果呈现在第五节;第六节对全文做了总结。

\section{被试和方法}
\subsection{被试}
一位19年前颈脊椎受过损伤\footnote{原文为 complete cervical spinal cord injury。}的男性被试(被试 S1)和一位健康的男性被试(被试 S2)参与了实验。两位被试分别在36岁和28岁有过BCI验的经历。

\subsection{实验范式}
我们用一套类似于文献[1]中的方法来记录和标记数据。展示给被试的是一个显示在笔记本显示器上的装有字母和数字1$-$9的6x6矩阵,该矩阵的行和列以100毫秒的间隔交替闪烁,闪烁经过\immature{block randomize}处理,经过12次闪烁后,每一行和每一列都恰好被点亮一次。

在每次实验中,被试被要求计数目标字符在矩阵中闪烁的次数。目标字符由操作员描述,并出现在屏幕的下方。每次实验闪烁的次数由系统随机地在$9\times 12, 10\times 12$或$11\times 12$中选择。为了观察被试的\immature{表现},在每次实验之后都会有一小段休息时间,被试要向操作员汇报他们计数的结果。

\!{每位被试都在两天内进行了测试并记录了两组数据集},在第一轮实验测试中,被试要拼写法语单词 ''lac'', ''nuage'', ''montagne''和''soleil''。在第二轮实验中,被试则拼写了 ''fromage'', ''chocolat'', ''pain''和''vin''。

\subsection{数据采集和预处理}
数据由 Biosemi Active 2 系统在2048Hz采样频率下采样 Fp1, Fp2, AF3, AF4, F7, F3, F4, F8, FC1, FC5, FC6, FC2, T7, C3, Cz, C4, T8, CP1, CP5, CP6, CP2, P7, P3, P4, P8, PO3, PO4, O1, Oz, O2 通道下采样获得。\!{epochs}
\subsection{伪迹的去除}
为了消除伪迹,样本的每一段\!{epoch} 的绝对值被计算出来。然后,每一段\!{epoch} 的最大绝对值被选为这个\!{epoch} 的特征值,这些数据以降序排列,其中前5\% ,也就是绝对值很大的\!{epoch} 会被舍弃。

\section{Boosting 最小二乘法}
Boosting 方法被用来根据训练数据计算出一个用来检测 P300 的函数。特别是 gradient boosting 被用于逐步最大化逻辑斯蒂克回归模型的\!{Bernuolli log likelihood}。文献[3],[4]中描述方法中\!{Bernuolli log likelihood} 的逐步最大化和[2]中的回归树算法是弱学习算法。本文则使用最小二乘法回归作为弱学习算法,这是由以下因素所决定的:
\begin{itemize}
  \item 使用最小二乘法我们得到一个易于理解和分析的离散函数$F$:它是自发脑电样本的简单线性组合;
  \item 相比回归树算法,最小二乘法可以节省更多计算资源,因此用于训练分类器的时间可以大为缩短;
  \item 先前的实验表明,应用于噪声非常大的数据集上时回归树算法可能较最小二乘法在\!{归一化错误}上有轻微优势。然而在特定信噪比的脑电下,最小二乘法的表现较回归树算法更优。
\end{itemize}

我们来详细讨论一下 boosting 方法和最小二乘法。我们将第$m$步分类器的集合记为$F_m$,将训练数据记为$X={x_i\in \mathbb{R}^K, i=1\ldots N}$,将对应的分类标签记为$Y={y_i \in {0,1}, i=1\ldots N}$。$K=C\times S$表示特征的数量,其中的$C$是自发脑电的通道数量,$S$则是每一个\!{epoch}的样本数量。逻辑斯蒂克回归模型如下所示:

\begin{equation}
p_m(y_i=1|x_i)=\frac{e^{F_m(x_i)}}{e^{F_m(x_i)}+e^{-F_m(x_i)}}
\end{equation}

$F_m$的\!{Bernuolli log-likelihood\footnote{伯努利对数似然函数}}如下式所示:

\begin{equation}
L(F_m; X, Y)=log(\prod\limits^N_{i=1}{p_m(y_i=1|x_i)^{y_i}p_m(y_i=0|x_i)^{1-y_i}})
\end{equation}

该似然函数在$F_0:=0$,$F_m=F_{m-1}+f_m$

\end{document}
