\documentclass[a4paper]{ecusttrans}

\eAuthor{Ulrich Hoffman, Gary Gacia, Jean-Marc Vesin, Karin Diserens and Touradj Ebrahimi}
\cStudent{王翔}
\cStudentClass{电自091}
\cStudentNo{10094309}
\cTitle{一种基于Boosting的用于脑机接口的P300检测方法}
\cKeywords{Boosting, 脑机接口, 脑电图, P300, 最小二乘法}

\begin{document}
  \setlength{\parindent}{2em}
  \sloppy
  \CJKtilde

  \begin{cAbstract}
  Gradient boosting~是一种基于许多弱分类器来建立一个强分类器的机器学习算法。本文主要阐述了一种建立在~gradient boosting~之上的,用于检测脑电图~(EEG)~中的事件相关电位~(ERP)~的算法。该算法通过检测人类脑电图中的P300电位来建立一种脑机接口~(BCI)~\footnote{也有翻译为{“自发电位”}的},一个拼写装置。本文所述方法的重要特性在于其分类的准确性及其概念的简单性。本算法经过一组由本实验室记录的数据集以及一组2003年BCI竞赛中所用到的测试数据集的测试。P300~拼写范式实验中的准确率在90\%到100\%之间。值得注意的是,本算法完全正确地完成了所有出自BCI竞赛测试数据集的字符推断任务。

  \end{cAbstract}

  \wuhao
  \section{导论}
  P300是一种人类脑电图中的特征波形,一般作为对小概率刺激事件的响应出现。经典的~Oddball~范式实验常常用来诱发P300:呈现给被试的是随机顺序的两类刺激,其中一类刺激为小概率事件,被试则被要求判断刺激属于哪一类。

  L.A.Farwell~和~E.Donchin~曾首创性地用~Oddball~范式实验来建立脑机接口。他们所用的方法中,呈现给用户的是一个装有字符的6x6的矩阵,该矩阵的行和列会按照随机顺序点亮。被试可以通过计数目标字符的点亮此书来从矩阵中选择一个字符。目标字符每点亮一次,就诱发一个~P300~波形,该~P300~波形可以通过适当的算法检测到。

  在本文中,我们阐述了一种简单却强大的方法来从脑电图中检测~P300,并用该方法来建立一个基于~P300~的拼写装置。我们使用了~gradient boosting~和最小二乘法相结合的方式来建立P300检测器。

  Gradient boosting~和最小二乘法的组合是一种有意思的检测~P300~的方法,因为这对组合有如下特性:

  \begin{itemize}
    \item 该算法通过一种朴素的方式建立线性分类规则,因此将分类器套用到新的数据上只需要少量操作,且可以实时地对~EEG~进行分类操作。此外,分类规则也很非常易于理解,也就是说,从分类规则可以很容易地看出哪些样本和哪些通道对于检测~P300~比较重要。
    \item 在分类准确度上,本文所所述方法和当前最先进的方法相比效果更为显著。以2003年~BCI~竞赛的测试数据为例,gradient boosting~略优于优胜方案。
    \item 复杂的优化算法,例如支持向量机或者独立组件分析方法,对本文所述方法而言都不是必需的。因此本算法具有易于实现、使用和扩展的特性。
  \end{itemize}

  本文余下部分的主要安排如下:第二节主要阐述本实验所使用的实验范式、被试和数据处理;第三节主要阐述~boosting~算法;第四节则主要阐述分类器的输出是如何用于推断被试选择的字符的;实验结果呈现在第五节;第六节对全文做了总结。
  \section{被试和方法}
\end{document}
